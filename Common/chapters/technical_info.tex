% !TEX root = ../main.tex

\chapter{Technical information}
\label{ch:technical_info}

\section{NIfTI}

\subsubsection{Origin of the format}
The Neuroimaging Informatics Technology Initiative (NIfTI) file format is the successor of the ANALYZE file format. The main problem of the latter was “the lack of adequate information about orientation in space, such that the stored data could not be unambiguously interpreted” .  For instance, there was a real confusion to determine the left and right sides of brain images.  Hence, the NIfTI file format was defined to obviate this major issue.


\subsubsection{Organization of data}
Unlike the ANALYZE format that used two files to store the meta-data and the actual data, the NIfTI file format stores them in one single file “.nii” but keeps this split between the real data and the header for compatibility. This has the advantage to facilitate the use of the data and avoid storing the data without the meta-data. The NIfTI format can also be compressed/decompressed on-the-fly using the “deflate”  algorithm.


\subsubsection{Overview of the header structure}
With the goal of preserving the compatibility between the “analyze” and the NIfTI format, both headers have the same size of 348 bytes. “Some fields were reused, some were preserved, but ignored, and some were entirely overwritten”. Details about the different fields contained in the header can be found here: \url{https://brainder.org/2012/09/23/the-nifti-file-format/}


\section{DICOM}

\subsection{Origin}

The acronym DICOM stands for Digital Imaging and Communications in Medicine. Before the 1980’s, images resulting from CT scans and MRIs were only decodable by machine manufacturers, while the medical community needed to export and share them for other tasks. For that reason, the ACR (American College of Radiology) and the NEMA (National Electrical Manufacturers Association) created a committee to build a standard. After two iterations with other names, DICOM was created in 1993. It standardized the representation of medical images and their transmission since it provided a network protocol built on top of TCP/IP.


\subsection{Data format}

DICOM files can be viewed as containers of attributes, also called tags. The values of the pixels themselves are stored under the "Pixel Data" tag. Every single DICOM file usually represents a 2-dimensional image, which will form a 3-dimensional volume when put all together. 

Other useful information such as the patient name and ID is directly stored within the DICOM files. This approach aims at linking each image to a specific person and event in order not to mix them up. Each DICOM file can be seen as part of a bigger dataset. 


\subsection{Processing images}

When manipulating DICOM files, multiple details must be taken into account. 

\subsubsection{Order}
First of all, the name of the files within datasets is a 6 digit number, from 000000 to the number of images minus one. However, this order doesn’t match the real order of the images. In fact, the correct order is given by the "Instance Number" tags contained in the various files. Therefore, converted images must be sorted by instance number. 

\subsubsection{Data manipulation}
Then, CT and MRI machines, as well as monitors, differ from one manufacturer to the other and even from one model to the other. DICOM takes this problematic into account by providing specific tags that allow to display the exact same representation of the data, no matter the hardware used. Otherwise, physicians may struggle to detect anomalies because of color and exposition-related variations. 
Therefore, before displaying or converting an image to any format (such as png), pixel data must be normalized. 

If the tags "Window Width" and "Window Center" (one always come with the other) are missing, a simple conversion is sufficient. The parameters to use to convert the data are given by two tags: 
\begin{itemize}
	\item Bits allocated: the number of bits used to represent a single pixel (value: 1 or a multiple of 8)
	\item Samples per pixel: the number of channels for each pixel

\end{itemize}

Examples: 
\begin{itemize}
\item 1 bit, 1 sample: black and white
\item 8 bits, 1 sample: grayscale
\item 8 bits, 3 samples: RGB
\item 16 bits, 1 sample: grayscale

\end{itemize} 

Otherwise, a linear transformation must be done to convert the stored representation of the pixels to the correct visualizable one. To do this, two step are required: 

\begin{enumerate}
	\item \textbf{Apply the Hounsfield correction}\newline
	Hounsfield Units (HU) are used in CT images it is a measure of radio-density, calibrated to distilled water and free air. The correction is calculated thanks to the following formula:
\begin{equation}
	HU = m * P + b
\end{equation}
	where~$m$ is the rescale slope,~$P$ the pixel value, ~$b$ the rescale intercept.

	\item \textbf{Apply a linear interpolation}\newline
	The result of the first operation then goes through a linear interpolation to get the right pixel values, based on the following conditions: 

\begin{equation}
\textrm{if } (P \leq c - 0.5 - \frac{w-1}{2}) \textrm{, then }y = y_{min}
\end{equation}

\begin{equation}
\textrm{else if } (P > c - 0.5 + \frac{w-1}{2}) \textrm{, then }y = y_{max}
\end{equation}

\begin{equation}
\textrm{else } y = (\frac{P - (c - 0.5)}{w-1}) + 0.5) * (y_{max} - y_{min}) + y_{min}
\end{equation}

where~$c$ is the window center,~$w$ window width, ~$P$ the pixel input value, ~$y$ the pixel output value, ~$y_{min}$ the minimal value of the output range, ~$y_{max}$ the maximal value of the output range.


\end{enumerate}


\subsubsection{Export}
Medical data can be represented over 16 bits, while exporting images to PNG require 8-bit images. To transpose 16 bits to 8, the following procedure can be applied:

\begin{enumerate}
	\item Find maximum value of the current image $Pixel_{min}$
	\item Find minimum value of the current image $Pixel_{max}$
	\item Pixel normalization: $Pixel = \frac{(Pixel - Pixel_{min}) * 255.0}{Pixel_{max} - Pixel_{min}}$
	\item Modify object type to 8 bits
	\item Export to PNG
\end{enumerate}