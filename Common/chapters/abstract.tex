% !TEX root = ../main.tex
Automating the detection of cancer contributes to an early detection and treatment, which increases the chances of recovery. Recent algorithms in artificial intelligence relying on deep learning have shown promising results in this field. Indeed, the usefulness of convolutional neural networks (CNNs) for segmentation or classification tasks is no longer to be proven. However, the performance of these models is often limited by the amount of data which is available to train the algorithm.

This thesis first presents a state-of-the-art convolutional neural network for pro\-state lesion classification. All the steps from the data processing to the smallest detail regarding the neural network training are explained, ensuring a complete reproducibility of the experiment. This model is then evaluated on the official SPIE-AAPM-NCI Prostate MR Classification Challenge dataset, achieving an AUC of 0.76. This result constitutes a solid baseline and confirms the proper functioning of the implementation.

On top of this implementation, a new transfer learning approach using lesions of multiple body parts (brain and lung) is built. This method shows that integrating information from diverse datasets improves automated prostate cancer diagnosis. Indeed, it appears that cancerous lesions coming from various body parts share low-level features that can be used to increase the generalization ability and performance of the prostate lesion classifier. This technique provides a concrete solution to the lack of available data for prostate classification and suggests that many other types of cancers can be taken advantage of. Thanks to this technique, the AUC achieved on our test set increases by 18\% (from 0.68 to 0.80). 
