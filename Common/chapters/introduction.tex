% !TEX root = ../main.tex

\chapter{Introduction}
\label{ch:introduction}


\setlength{\marginparwidth}{3cm}\leavevmode \marginnote{\textbf{Julien}}According to the World Health Organization, cancer caused $9.6$ million deaths in 2018, making it the second leading cause of death~\cite{44}. An early detection and treatment increases the chances of recovery. In this context, Computer-Aided Diagnosis (CAD) systems can play a massive role by preventing health professionals from missing positive diagnoses. Thanks to the attention that deep learning has gotten over the last decade, newer and better diagnosis systems have transfered the advantages of deep learning to cancer detection, diagnosis and localization tasks.\\
Deep learning applications require a lot of data to perform well. While certain fields profit from massive amounts of publicly available data, the medical field is quite the opposite. First of all, medical information is protected by the doctor-patient confidentiality and cannot be shared freely. As a consequence, data first has to be collected, organized and anonymized. Then, additional information regarding the clinical significance of the samples, the location of the lesions, etc. must be provided so that deep learning models can make use of the data. This whole process is time consuming and medical institutions don't always see the benefits which could ensue from publishing datasets of good quality. As a consequence, the lack of data is one of the toughest challenge related to this field. In order to overcome it, techniques such as transfer learning exist. The latter aims at using independent but similar datasets in order to increase the performance of a model on a target dataset. In other words, models are trained on the former in order to learn relevant features which improve the results on the latter. In the case of cancer detection and classification, only few datasets are available for each body part. Therefore, the idea behind this work is to make use of datasets of different body parts to improve the classification of prostate lesions.\\
This work is divided into various parts. First, the related literature was reviewed in order to gather techniques and architectures which gave decent results.\\
Second, deep learning is presented from the ground up under the historical and technical points of view. This section demystifies the topic by making an overview of the mathematical concepts and definitions related to it, which makes the understanding of the technical part possible without any background in the field.\\
The next chapter deals with medical knowledge related to cancer. Characteristics of the disease are presented, before focusing on the various file formats which medical imaging data are exported into. The last part is critical since a lot of data processing was performed in order to be able to make use of these files.\\
Then, the architecture presented in a prostate cancer classification research paper was reproduced. This chapter allows to set a baseline proving that the methods and implementations work, from the image preprocessing to the neural network.\\
Finally, the last part is dedicated to transfer learning and its application to cancer classification.


% The most lethal kind is the lung cancer ($1.76$ million deaths in 2018).

\section{Motivation}

\setlength{\marginparwidth}{3cm}\leavevmode \marginnote{\textbf{Johan}}Cancer is a serious concern in today's society. An early detection is key to maximize the chances of recovery. As explained by Nicholas Petrick~\cite{50}, "it has long been recognized that clinicians do not always make optimal use of the data acquired by an imaging device. The limitations of the human eye-brain system, limitations in training and experience, and factors such as fatigue, distraction, and satisfaction of search may all contribute to suboptimal use of available information". Under these circumstances, CAD systems can contribute to address some of these issues by being used as an aid for clinicians. A complete CAD system is composed of two parts. The first one is called CADe for "computer-aided detection". This part includes medical  image  analysis  tasks  such  as  segmentation,  identification,  localization  and  detection. The other part is CADx for "computer-aided diagnosis" which aims at extracting the characteristics of lesions to classify them according to their malignancy. CAD systems provide multiple advantages.\\
First of all, used as an extra diagnosis, they decrease the probability of missing positive diagnoses. Then, they can also speed up the diagnosis process by proposing regions of interest to clinicians, which, in time, could reduce the screening costs.\\
Similarly, lower screening costs imply a greater accessibility to screening tests, which can result in an earlier cancer detection. As stated before, detecting cancers at an early stage maximizes the probability of curing them. Another benefit of using CAD systems, especially the ones based on deep learning, is the number of cases that the latter are trained on. In fact, in order to built efficient deep learning models, large amounts of data are required. Consequently, efficient CAD systems usually see a lot more cancer cases than beginner clinicians.\\
However, these advantages remain partially theoretical nowadays. Research is still ongoing, aiming at improving existent CAD systems. Problems like poor detection rates, small amounts of available data and bad levels of generalization are still recurrent. Therefore, this situation can and must be improved.

%Consequently, this thesis presents how to build a modern Deep Learning model for prostate lesions classification and proposes a model based on transfer learning to overcome the lack of available data in the medical field.



\section{Contributions}

\setlength{\marginparwidth}{3cm}\leavevmode \marginnote{\textbf{Johan}}This work proposes a state-of-the-art deep learning model for prostate lesion classification and proposes a method based on transfer learning using different body parts to overcome the lack of publicly available data in the medical field. Concretely, the major contributions of this work are:
\begin{itemize}
\item A state-of-the-art deep learning model for prostate cancer detection. All steps are clearly described, the corresponding scripts are available on GitHub\footnote{\url{https://github.com/eXascaleInfolab/2019_Hospital-Fribourg}} and the raw results are presented. This makes the experiment completely reproductible. 

\item A transfer learning pipeline that makes use of different body parts to increase the classification performance on one chosen body part. This technique is a concrete solution to overcome the lack of available data for each body part. Furthermore, it increases the generalization ability of the neural network.


\item Processing scripts for the SPIE-AAPM-NCI Prostate MR Classification Challenge dataset, the Lung CT Challenge dataset and the Kaggle "Brain MRI Images for Brain Tumor Detection" dataset. This includes the conversion of DICOM and PNG files to NumPy arrays, their registration (alignment and resizing to the same resolution for stacking purpose, ensuring the same amount of tissue on each image), their augmentation and the organization of the resulting images into multiple subsets (training, validation and test) that can be used as input to machine learning algorithms. Also, the class imbalance problem is taken into account. The processing scripts automatically augment each class in order to balance the number of elements. In addition to this, an alternative relying on undersampling was implemented as a PyTorch sampler.

\item Visualization scripts for DICOM, NIfTI and RAW medical file formats. A system to navigate through 4D data (width, height, depth and time) using the directional arrows is implemented: left/right arrows to vary the time axis and up/down to navigate through the difference slices (i.e the depth) of all images of a patient. 3D data is also supported with the same functionalities (apart from navigating through the fourth dimension (time)). 


\item Processing scripts for the PROSTATEx challenge. This goes from the preprocessing of the challenge test images to the generation of the CSV file containing the predictions of a given model, which are probabilities $[0,1]$ of the lesion being malignant ("benign lesion" if $< 0.5$, "malignant tumor" otherwise).

\end{itemize}
Other minor contributions:
\begin{itemize}
\item Verification scripts to check the gradient flow of a neural network, the cropping of images, the presence of NaN values in images, etc. All these scripts can easily be used in other projects.
\end{itemize}


\section{Work repartition}

This thesis was jointly written by two authors. The name(s) of the contributor(s) in the margins of the following chapters only concern the written thesis. Both of them contributed equally to the work leading to the writing of the latter. The work was either done together or in parallel, depending on the tasks. In any case, parts written by one of the authors were reread by the other. Finally, the order in which the names are cited does not represent the quantity of work performed and was chosen arbitrarily. 